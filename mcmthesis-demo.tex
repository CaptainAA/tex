%%
%% This is file `mcmthesis-demo.tex',
%% generated with the docstrip utility.
%%
%% The original source files were:
%%
%% mcmthesis.dtx  (with options: `demo')
%%
%% -----------------------------------
%%
%% This is a generated file.
%%
%% Copyright (C)
%%       2010 -- 2015 by Zhaoli Wang
%%       2014 -- 2019 by Liam Huang
%%       2019 -- present by latexstudio.net
%%
%% This work may be distributed and/or modified under the
%% conditions of the LaTeX Project Public License, either version 1.3
%% of this license or (at your option) any later version.
%% The latest version of this license is in
%%   http://www.latex-project.org/lppl.txt
%% and version 1.3 or later is part of all distributions of LaTeX
%% version 2005/12/01 or later.
%%
%% This work has the LPPL maintenance status `maintained'.
%%
%% The Current Maintainer of this work is Liam Huang.
%%
%%
%% This is file `mcmthesis-demo.tex',
%% generated with the docstrip utility.
%%
%% The original source files were:
%%
%% mcmthesis.dtx  (with options: `demo')
%%
%% -----------------------------------
%%
%% This is a generated file.
%%
%% Copyright (C)
%%       2010 -- 2015 by Zhaoli Wang
%%       2014 -- 2019 by Liam Huang
%%       2019 -- present by latexstudio.net
%%
%% This work may be distributed and/or modified under the
%% conditions of the LaTeX Project Public License, either version 1.3
%% of this license or (at your option) any later version.
%% The latest version of this license is in
%%   http://www.latex-project.org/lppl.txt
%% and version 1.3 or later is part of all distributions of LaTeX
%% version 2005/12/01 or later.
%%
%% This work has the LPPL maintenance status `maintained'.
%%
%% The Current Maintainer of this work is Liam Huang.
%%
\documentclass{mcmthesis}
\mcmsetup{CTeX = false,   % 使用 CTeX 套装时,设置为 true
        tcn = 2016619, problem = D,
        sheet = true, titleinsheet = false, keywordsinsheet = true,
        titlepage = true, abstract = true}
\usepackage{newtxtext}%\usepackage{palatino}
\usepackage{lipsum}
\usepackage[centertags]{amsmath}
\usepackage{mathrsfs}
\usepackage{graphicx}
\usepackage{float}


\begin{document}
\begin{abstract}
This is my first experience for take part in the mcm contest. 

\begin{keywords}
keyword1; keyword2
\end{keywords}
\end{abstract}
\maketitle
%% Generate the Table of Contents, if it's needed.
\tableofcontents
\newpage

%%Generate the Memorandum, if it's needed.
\memoto{\LaTeX{}studio}
\memofrom{Liam Huang}
\memosubject{Happy \TeX{}ing!}
\memodate{\today}
\logo{\LARGE I'm pretending to be a LOGO!}
\begin{memo}[Memorandum]
\lipsum[1-3]
\end{memo}
%%
%%
%%
\section{Introduction}
\subsection{Background}
In contemporary society, people tend to emphasize the
importance of the teamwork since team work much effective
than the individual. This is undeniable to say that teamwork 
plays an indispensible role of the human achievement.
The collaboration within a group of people can help to explore the 
potential of each member and the strong union built between the team members 
make them competitive. One of the most general application of the teamwork 
is the competitive team sports, which indicates the strength of the cooperation.
The coach of the Huskies is seeking for a model that can 
formalize and quantify the dynamic features leading their 
team to success. We, the Intrepid Champion Modeling team,
trying to find a solution, are asked to help to find out the
optimal strategy for the Huskies coach.  

\subsection{Task at hand}
\begin{enumerate}
  \item Construct a ball passing network within the players,
  and also define the basic indicators and properties of network
  \item Figure out the important performance indicators leading to a
  successful teamwork, including the team level process. and find out 
  feasible strategies when confronted with different opponents
  \item using the results of our analysis to suggest the Huskies coach
  that what strategies should be taken to improve the performance 
  \item derive a more general method to build up an effective teams based
  on the current model, figure out additional indicators needed to 
  perfect the model
     
\end{enumerate}

\subsection{assumptions of the models}

xxxxxxxxxx
\begin{itemize}
\item the angular velocity of the bat,
\item the velocity of the ball, and
\item the position of impact along the bat.
\end{itemize}

%%\emph{center of percussion} [Brody 1986], 

%%\begin{Theorem} \label{thm:latex}
%%\LaTeX
%%\end{Theorem}
%%\begin{Lemma} \label{thm:tex}
\TeX .
%%\end{Lemma}
%%\begin{proof}
%%The proof of theorem.
%%\end{proof}

\subsection{Other Assumptions}
xxxxxxxxxxx

\section{Model assumption}







\section{The basic model of social network analysis}
\subsection{introduction of the model}
To create a network to analyze the interaction between the player,
we introduce the social network analysis to construct our model.
Social network analysis, focused on uncovering the pattern of
people's interaction, can shed light on how individual actions create social structure
From the perspective of social network, human interaction in social
environment can be described as a pattern or rules based on their relationship, 
and the regular pattern based on this relationship embodies the social structure.
The quantitative analysis of this structure is radical departure of social 
network analysis.

The reason why we choose the social network to analysis the performance of 
football team is that the cooperation within the players is pretty
assemble the interaction between the members in the society.
Generally speaking, network is a set composed of nodes and relationships between nodes.
Social network is also a set being formed of social actors(taken as nodes) and their relations.
Football is a sport of collective antagonism. Due to the size of the field, passing is considered 
to be the most critical method for the attacking team to set up an offence and create chances 
for scoring. Therefore, passing is the most important technique used in football matches.
For the attacking team, the running route of the ball in the passing process is like a 
network fiber to connect each player and each line organically. Through the transfer of 
the ball, players establish a relationship based on the passing of the ball with each other. 
The more passes players pass, the closer the relationship becomes. The 
passing relationship between players can be viewed as a kind of "social network". 
In this "network", players are "network nodes", and the passing between players 
can be regarded as their "relationship". Players (nodes) and passing (ties) 
constitute a social network in football.  
In a nutshell, we create a network for the ball passing between players, where each 
player is a node and each pass constitutes a link between
players. To analyze the network, we using the software ucinet to Construct
the graph of the network of each games. Besides we take the time span in to
consideration by analyzing the data from the first, second half and full time.   


\section{calculating and analysis of the model}
\paragraph{\textbf{brief introduction}}
In our model, we analysis the network of Huskies from three perspectives. We use the
software \textit{ucinet} to draw the graph of network and the data comes from the 
given data set.
In the social network analysis, the type of the network can be divided into tree parts:
the monolithic network, local network and the individual network. The monolithic network analysis
provides an overlook of the Huskies social network, while the local network provides a view of
cooperation between small group players like dyadic and triadic configurations and from the individual
network, we gain an insight into the role of individual players.

\subsection{monolithic analysis}
\subsubsection{definition and conception}
\paragraph{\textbf{Adjacency matrix}}

It describes adjacent relationship among nodes(also players in this
context), usually notated as a square matrix A=($a_{ij}$)$_{n*n}$.
Usually, the main diagonal elements of A are all 0. The value of $a_{ij}$is
the number of passes that player $v_{i}$ passes it to player $v_{j}$
during certain amount of time.
The sum of all elements in the i-th row is the out-degree $k_{i}^{out}$
of player (node) $v_{i}$ (the number of links beginning with node
$v_{i}$). That is to say $\sum_{j=0}^{n}a_{ij}=k_{i}^{out}$, i=1,2,3,...,n
The sum of all elements in the j-th columns is the in-degree $k_{i}^{in}$
of player (node) $v_{j}$ (the number of links ending up with node
$v_{j}$). That is to say $\sum_{i=0}^{n}a_{ij}=k_{j}^{in}$, j=1,2,3,...,n
Denote $k_{i}$ as the degree of player(node) $v_{i},$$k_{i}=k_{i}^{out}+k_{i}^{in}$, i=1,2,3,...,n

We constructs the adjacent matrix by using the python, the code is attached in the appendix.

\paragraph{\textbf{original matrix}}
It is the transformation of the Adjacency matrix, which is symmetric. We can get the passing times
of players intuitively from the original matrix. We generates 229 original matrix 

\begin{figure}[h]
  \small
  \centering
  \includegraphics[width=13cm]{ownfigure/passingeventM1.png}
  \caption{original matrix (e.g.M1)} 
  \end{figure}

We construct the 229 original matrixs using the pivot_table function in the Excel, the
approach is also attached in the appendix.

\paragraph{\textbf{Density}}

\noindent The ratio of the actual number of links to the maximum number of links
that can exist in the network.

\noindent For the directed network, denote $de$ as the density of it:

\[
de=\frac{m}{n(n-1)}
\]

\noindent where m is the number of actual and directed links in the network(e.g.A->B
and B->A are two different links), n is the number of players(nodes)
on the field(network).

\paragraph{\textbf{Average distance}}
\noindent For each pair of nodes, the algorithm finds the number of edges in
the shortest path between them.

Denote d$_{ij}as$ the distance between player (node) v$_{i}$ and
v$_{j}$,which means the minimum number of intermediaries required
for v$_{i}$ and v$_{j}$to get in touch.

Denote D as average distance of the network, we have:

\[
D=\frac{2\sum_{i<j}d_{ij}}{n(n-1)}
\]
where n is the total number of players of one team on the field.Since
d$_{ij}=d_{ji}$, we only add those d$_{ij}$ with i<j.

In the calculation of $d_{ij}$, we use BFS(Breadth First Search),
which is a graph search algorithm that begins at the root node and
explores all the neighboring nodes. Then for each of those nearest
nodes, it explores their unexplored neighbor nodes, and so on, until
it finds the goal, by ``Network→Cohesion→Distance''in $ucinet6.2$.
The schematic diagram of BFS is shown as follow:

\begin{figure}[h]
  \centering
  \includegraphics[width=8cm]{ownfigure/Breadth first search.png}
  \caption{BFS}
\end{figure}

The average distance is used assess the strong ties between the players
and effectiveness of passing the ball.

\paragraph{\textbf{Passing rate}}

\noindent The ratio of the number of Huskies' passes to the number of opponent's
passes, denoted as P

\[
P=\frac{P_{H}}{P_{T}}
\]

where $P_{H}$is the number of Huskies' passes,and $P_{T}$is the
number of opponent's passes

\subsubsection{analysis of the monolithic network}

To give the social network an overview, we use monolithic approach to analyze the team network.
In order to reflect the dyadic passing configuration of players, it is necessary to transform
the original matrix into a symmetric matrix based on the sum of the out-degree and in-degree.
By using the \textit{ucinet} (the operation is attached in the appendix) to plot the network of passing ball, we derive the passing track
net work of the Huskies(take the match M1 as an example).

\begin{figure}[h]
\small
\centering
\includegraphics[width=8cm]{ownfigure/主队M1.jpg}
\caption{Passing track net work(e.g.M1)} 
\end{figure}

The passing track network visually describe the interaction between the Huskies players.
In games 1, the more passes a player generates (node degree), the larger the circle of nodes.
The connection between nodes represents the number of players direct passes, and the more passes,
the thicker the connection. 

Apparently, the passing track network only provides an overview of the team structure.
So, index of the network is used to quantify the connection between the 
network members. We use the Index like Density, Average Distance, Total Number Of Huskies‘ Passingevents,
Passing Ratio, OwnScore, Outcome to overall assess the monolithic construction of the Huskies.
The graph is shown below:
\begin{figure}[h]
  \large
  \centering
  \includegraphics[width=8cm]{ownfigure/整体网络数据.png}
  \caption{Monolithic network index} 
  \end{figure}
      
Deriving from the table, the average distance larger than 1 means that almost every members has passed ball
to each other in series of games, given us a rough glimpse of the overall connections between each members.

\subsubsection{chi-squre test of the distence index}
To identify whether there are actually relationship between the indexs of the monolithic network and their success.
We choose the four index and use the chi-square independence test to exam the correlation between two variables.
We set the confidence value alpha = 0.05, and setting the judgement values \textit{re}. If the outcomes of r equals 1, we
choose to reject assumption, otherwise reject it. And the scatter diagrams below show the test result derives from python.
And the 
\begin{figure}[H]
  \centering
  \subfigure{
    \label{fig:subfig:a} 
    \includegraphics[width=0.2\textwidth]{chi-square/chi1.png}
    \includegraphics[width=0.2\textwidth]{chi-square/chi2.png}
    }\\
  \hspace{0in}
  \subfigure{
    \includegraphics[width=0.2\textwidth]{chi-square/chi3.png}
    \includegraphics[width=0.2\textwidth]{chi-square/chi4.png}
    }\\
  \caption{chi-square test of Density, Average distance, Number_of_passingevents, Ratio_of_passingevents}
\end{figure}

As our expected, all of the four indexs pass the tests, which means that there are quite correlation
between the monolithic indexs and success of the game. It supports that indexs of the monolithic network have 
impact on the success of games.

\subsection{Local analysis}
The chi-square test shows that the indexs of the monolithic network is adapted to most of the situation.
However, monolithic index of the Match4 is abnormal since although performance of the monolithic indexs are \
pretty well, but they still lose the games. 

\begin{figure}
  \centering
  \includegraphics[width = 10cm]{ownfigure/M4.png}
 \end{figure}

To have a more profound understanding of Huskies social network, we introduce the Local network to analysis
the cooperation constructions.

\subsubsection{definition and conception}
\paragraph{\textbf{Centralization}}

Centralization can characterize the centrality of the whole group
and measure the degree of concentration to the center, so it can be
taken as the estimation that to what extent the network depend on
the minority. Centralization includes Network Centralization and Network
Centralization Index.

\subparagraph{\textbf{Network Centralization}}

\noindent Network centralization is a network index that measures the degree
of dispersion of all node centrality scores in a network from the
maximum centrality score obtained in the network,denote as $C$

\[
C=\frac{\sum_{i=1}^{n}(C_{max}-C_{i})}{max\left[\sum_{i=1}^{n}(C_{max}-C_{i})\right]}
\]

where $C_{max}$is the largest centralization degree of those nodes
on the network(players on the field),$C_{i}is$ the centralization
degree of node(player)$v_{i}$.

denote C_${AD}$ as the absolute network centralization

\[
C_{AD}=\frac{\sum_{i=1}^{n}(C_{AD_{max}}-C_{AD_{i}})}{n^{2}-3n+2}
\]

where $C_{AD_{max}}$is the largest absolute centralization degree
of those nodes on the network(players on the field),$C_{AD_{i}}is$
the absolute centralization degree of node(player)$v_{i}$,and n is
the total number of players of one team on the field.

\subparagraph*{\textbf{Network Centralization Index}}

\noindent denoted as $C_{B}$
\[
C_{B}=\frac{\sum_{i=1}^{n}(C_{AB_{max}}-C_{AB_{i}})}{(n-1)^{2}(n-2)}
\]

where $C_{AB_{max}}$is the largest absolute node betweenness of those
nodes on the network, $C_{AB_{i}}$is the absolute betweenness centrality
of node $v_{i},$and n is the total number of players of one team
on the field.
%%图的位置不对
\paragraph*{\textbf{Core-periphery Analysis}}
Core-periphery construction is a highly centralized groups but sparse
outsize the center formed by connections between several elements in the network.
We analyze the network by dividing the nodes into two groups,core and periphery,
depends on the importance and sparsity of the nodes.
\begin{figure}[h]
  \centering
  \includegraphics[width = 8cm]{ownfigure/zhongxin.png}
  \caption{Core-periphery Analysis}
 
\end{figure}


\paragraph*{\textbf{Betweenness Centrality}}

\noindent Denote the betweenness centrality of node(player)$v_{i}$as$C_{AB_{i}}$and
the ability of $v_{i}$to control the interaction between $v_{j}$ and
$v_{k}$(the possiblity that $v_{i}$ is located in geodesics between
j and k)as $b_{jk}(i)$:

\[
C_{AB_{i}}=\sum_{j}^{n}\sum_{k}^{n}b_{jk}(i),j\neq k\neq i,and j<k
\]

\[
b_{jk}(i)=\frac{g_{jk}(i)}{g_{jk}}
\]

where $g_{jk}$ is the number of geodesics between $v_{j}$ and $v_{k}$,
and $g_{jk}(i)$ is the number of geodesics between $v_{j}$ and $v_{k}$
which pass $v_{i}$.

\textbf{Local network analysis}
\subsubsection{the dyadic and triadic configurations}

Based on our previous analysis, the dyadic configuration is the connections
between the nodes, which is equals to the passing events of the members.
We using the \textit{highchart} to plot the connection between the palyers, using the data
from the MatchID_1.

\begin{figure}
  \centering
  \small
  \includegraphics[width = 10cm]{ownfigure/MATCHID1-passing.png}
  \caption{connections}
\end{figure}

The ties between the players denote the amounts of passing between the players.
The more the passing happens, thicks the ties are. The length of the arc in the circle
of one players denote the times of passing balls. Longer arc indicates that he creates
more pass. 

The triadic configurations is based in the analysis of the triple members group.
We use \textit{python} to exam all trio groups and calculate the passing events generated
by them. Then we sorted the trio groups by the values of pass(values over 20).

\begin{figure}
  \centering
  \small
  \includegraphics[width = 8cm]{ownfigure/trio.png}
  \caption{trio connection of collaboration}
\end{figure}

\subsubsection{Core-periphery Analysis}
However, viewing the Local network by quantifying the values of dyadic and triadic configurations
cannot solve the problems since it only provides the insight of team's cooperation. Since the 
Core-periphery analysis divides the social network into two distinct groups, it can give us 
a more comprehensive view of social network. Therefore, we should simplify the social net work
by search for the connection between the core players.
We use the \textit{Unicet} to draw the core-periphery matrixs as follows. We analysis the MatchID35 
for an example.

\begin{figure}
  \center
  \small
  \includegraphics[width = 12cm]{ownfigure/M35H1.png}  
  \includegraphics[width = 12cm]{ownfigure/M35H2.png}
  \caption{Core-periphery analysis of MatchID35}
\end{figure}

From the Core-periphery analysis, it is intuitive that the cooperation between the core 
players likes Defender , Defender , Forward . This generates a strong offense network,
which is critical to their success. 
In the passing net work of the Huskies, there are frequent pass happens between the core players and 
the other players. 
It is important to mention that the core players may not be the one that create the largest number
of passing but he should be the key players who can connect the periphery players and the core players.

Also, we use \textit{highchart} to plot the bubble diagram of the MatchID35(take the whole season as time span), the 

\begin{figure}
  \center
  \small
  \includegraphics[width = 8cm]{ownfigure/core.png}
  \caption{Core-periphery analysis for whole season}
\end{figure}
Same as the previous analysis, the core players have the largest amount of in-degree and out-degree.
It suggest that the definition of core-periphery reflects the importance of the core player.
%%还要多写


\subsection{individual network analysis}
To deeper the research, we choose to analysis of the every individual players and try to find out the
different role they play.
\subsubsection{definition and conception}
\paragraph{\textbf{Betweenness Centrality}}

Denote the betweenness centrality of node(player)$v_{i}$as$C_{AB_{i}}$and
the ability of $v_{i}$to control the interaction between $v_{j}$and
$v_{k}$(the possiblity that$v_{i}$ is located in geodesics between
j and k)as $b_{jk}(i)$:

\[
C_{AB_{i}}=\sum_{j}^{n}\sum_{k}^{n}b_{jk}(i),j\neq k\neq i,andj<k
\]

\[
b_{jk}(i)=\frac{g_{jk}(i)}{g_{jk}}
\]

where $g_{jk}$ is the number of geodesics between $v_{j}$ and $v_{k}$,
and $g_{jk}(i)$ is the number of geodesics between $v_{j}$ and $v_{k}$
which pass $v_{i}$.

\paragrapgh{\textbf{Node Centralilty}}
Node Centrality degree refers to the number of direct connections between
nodes and other nodes in the network, while the out-degree and 
in-degree are included in the directed network. If a node has the highest degree
, it is said to be at the center of the local network and it has authority.

\subsubsection{analysis of the individual network}
\paragrapgh{}
We analysis the individual network of the 












\begin{appendices}

\section{First appendix}

In addition, your report must include a letter to the Chief Financial Officer (CFO) of the Goodgrant Foundation, Mr. Alpha Chiang, that describes the optimal investment strategy, your modeling approach and major results, and a brief discussion of your proposed concept of a return-on-investment (ROI). This letter should be no more than two pages in length.

\begin{letter}{Dear, Mr. Alpha Chiang}

\lipsum[1-2]

\vspace{\parskip}

Sincerely yours,

Your friends

\end{letter}
Here are simulation programmes we used in our model as follow.\\

\textbf{\textcolor[rgb]{0.98,0.00,0.00}{Input matlab source:}}
\lstinputlisting[language=Matlab]{./code/mcmthesis-matlab1.m}

\section{Second appendix}

some more text \textcolor[rgb]{0.98,0.00,0.00}{\textbf{Input C++ source:}}
\lstinputlisting[language=C++]{./code/mcmthesis-sudoku.cpp}

 \end{appendices}


\end{document}
%%
%% This work consists of these files mcmthesis.dtx,
%%                                   figures/ and
%%                                   code/,
%% and the derived files             mcmthesis.cls,
%%                                   mcmthesis-demo.tex,
%%                                   README,
%%                                   LICENSE,
%%                                   mcmthesis.pdf and
%%                                   mcmthesis-demo.pdf.
%%
%% End of file `mcmthesis-demo.tex'.
